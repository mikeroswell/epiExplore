\documentclass[12pt]{article}
%% \usepackage{amssymb}

\newcommand{\dt}{dt\,}
\newcommand{\ds}{ds\,}
\newcommand{\dtau}{d\tau\,}
\newcommand{\drho}{d\rho\,}

\begin{document}

The Roswell conjecture is true.

The conjecture states that the “excess” (i.e., non-Poisson) variation of individual case reproductive numbers in the standard SIR model outbreak is 1.

We define $x$ as the proportion of the population susceptible, $y$ as the proportion infectious, and incidence $i = Bxy$.

Let $f$ be the distribution of residence times in the infectious compartment. This is exponential in the standard SIR and we will make that assumption later. Define $F$ as the survival distribution function 
$F(t) 
	= 1 -\int_{\tau<t} \dtau f(\tau)
	= \int_{\tau>t} \dtau f(\tau)
$.

Define the final size $Z = \int\dt i(t)$. Individuals are characterized by their infection time $\tau$ and recovery time $\rho$. The size of each such class is $w(\tau, \rho) = i(\tau) f(\rho)$. Their expected case reproductive number (which we will use to calculate the mean and the excess variance) is $C(\tau, \rho) = B \int_{\tau<t<\rho} \dt$.

Define the raw moments of $C$ as $C_k = \int\int_{\tau<\rho} \dtau\drho w(\tau, \rho) (C(\tau, \rho))^k$. We expect $C_0$ 
\end{document}

