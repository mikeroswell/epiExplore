\documentclass[12pt]{article}
%% \usepackage{amssymb}

\newcommand{\dt}{dt\,}
\newcommand{\ds}{ds\,}
\newcommand{\dtau}{d\tau\,}
\newcommand{\drho}{d\rho\,}

\newcommand{\intint}{{\int\!\!\int}}
\newcommand{\intintint}{{\int\!\!\int\!\!\int}}

\begin{document}

The Roswell conjecture is true.

The conjecture states that the “excess” (i.e., non-Poisson) variation of individual case reproductive numbers in the standard SIR model outbreak is 1.

Define $x$ as the proportion of the population susceptible, $y$ as the proportion infectious, and incidence $i = Bxy$.

Let $f$ be the distribution of residence times in the infectious compartment. This is exponential in the standard SIR and we will make that assumption later. Define $F$ as the survival distribution function 
$F(t) 
	= 1 -\int_{\tau<t} \dtau f(\tau)
	= \int_{\tau>t} \dtau f(\tau)
$ (these are equivalent because the full integral of the distribution $f$ is one.

Individuals are characterized by their infection time $\tau$ and recovery time $\rho$. The size of each such class is $w(\tau, \rho) = i(\tau) f(\rho-\tau)$. The expected case reproductive number (which we will use to calculate the mean and the excess variance) is $C(\tau, \rho) = B \int_{\tau<t<\rho} x(t) \dt$.

Define the raw moments of $C$ as $C_k = \intint_{\tau<\rho} \dtau\drho w(\tau, \rho) (C(\tau, \rho))^k$. 
We expect $C_0=Z$, where the final size $Z = \int\dt i(t)$. From there, we will calculate $\mu_C = C_1/C_0$, and the squared CV $\kappa_c = C_0C_2/C_1^2-1$. We expect $\mu_C=1$. Thus the conjecture is equivalent to $\kappa_C = 1$.

We have:
\begin{eqnarray}
	C_0
	&=& \intint_{\tau<\rho} \dtau\drho w(\tau, \rho) 
	\\ &=& \int \dtau i(\tau) \int_{\rho>\tau} \drho f(\rho-\tau)
	\\ &=& \int \dtau i(\tau), 
\end{eqnarray}
as expected.

Next:
\begin{eqnarray}
	C_1
	&=& \intint_{\tau<\rho} \dtau\drho w(\tau, \rho) c(\tau, \rho)
	\\ &=& \intintint_{\tau<t<\rho} \dtau\dt\drho 
		i(\tau) f(\rho-\tau) x(t)
\end{eqnarray}

\end{document}

