\documentclass[12pt]{article}
\usepackage{amssymb}

\begin{document}
\section{defining the case reproduction number}
Start with the normalized SIR:

$$ i = Bxy; \dot x = -i; \dot y = i-y $$

If $C(t)$ is the expected case reproductive number for the cohort infected at time $t$, then the overall mean is:

$$
	\bar C
	= \frac{\int{dt\, i(t) C(t)}}{\int{dt\, i(t)}}
	= \frac{\int{dt\, i(t) C(t)}}{Z},
$$

where $Z$ is the size of the epidemic.

Individuals in a cohort are effectively assumed to differ only in terms of how long they are infectious. We calculate $C(t)$ as a weighted average over this duration. The probability of having duration $D$ is just $\exp(-\delta)$ in this normalized model, and the reproductive number for this group is given by:
$$C(t, D) = B \int^D{d\delta\, x(t+\delta)}, $$

Since the probability of being infectious for exactly $D$ is $\exp(-D)$:
\begin{eqnarray}
	C(t)
	&=& B \int{dD\, \exp(-D) \int^D{d\delta\, x(t+\delta)}}
	\\ &=& B \int{dD\, \exp(-D) \int_{\delta<D}{d\delta\, x(t+\delta)}}
	\\ &=& B \int{d\delta\, x(t+\delta)\int_{D>\delta}{dD\, \exp(-D) }}
	\\ &=& B \int{d\delta\, x(t+\delta) \exp(-\delta)}
\end{eqnarray}

or, in terms of time of contact $\tau=t+\delta$:
$$C(t) = B \int_{\tau>t}{d\tau\, x(\tau) \exp(t-\tau)}, $$

We can also solve the $\dot y$ equation, by looking at who was infected at time $\theta$, and how many of them survived:

$$ y(t) = \int_{\theta<t}{d\theta\, i(\theta)\exp(\theta-t)},$$

and then expand:

\begin{eqnarray}
	Z \bar C
	&=& \int{dt\, i(t) C(t)}
	\\ &=& B \int{dt\, i(t) \int_{\tau>t}{d\tau\, x(\tau) \exp(t-\tau)}}
	\\ &=& B \int{d\tau\, x(\tau) \int_{t<\tau}dt\, i(t) \exp(t-\tau)}
	\\ &=& B \int{d\tau\, x(\tau) y(\tau)}
	\\ &=& \int{d\tau\, i(\tau)}
	\\ &=& Z
\end{eqnarray}

Thus, $\bar C = 1$, as expected. [Note: we could presumably fiddle with having a finite $y_0$ by playing with integral limits in the array above to get something like $(Z-x_0)/Z$.]

\section{between-cohort sum of squares expansion}
Now try to expand the sum of squares. One trick we're using here is that to compute the square of the integral of $f(x)$ over a range, we can just take twice the integral of $f(x)f(y)$ over a triangle where we assume we know $x>y$.

First, we look only at between-cohort variance, so our first step is to expand the sum of squares there.

\begin{eqnarray}
\int{dt\, i(t) C^2(t)}
	\\ = \int{dt\, i(t) \left(
		B \int_{\tau>t}{d\tau\, x(\tau) \exp(t-\tau)}
	\right)^2}
	\\ = B^2 \int{dt\, i(t) \left(
		\int_{\tau>t}{d\tau\, x(\tau) \exp(t-\tau)}
	\right)^2}
	\\ =  2B^2 \int{dt\, i(t)
		\int_{\tau>t}{d\tau\,  x(\tau) \exp(t-\tau)
			\int_{\sigma>\tau}{d\sigma\, x(\sigma)\exp(t-\sigma)}
		}
	}
	\\ =  2B^2 \int{
		d\sigma\, x(\sigma) \int_{\tau<\sigma}{
			d\tau\,  x(\tau) \int_{t<\tau}{
				dt\, i(t) \exp(t-\tau) \exp(t-\sigma)
			}
		}
	}
\end{eqnarray}

It is not obvious how to do the inner integral, and we expect it to be complicated. It's also not obvious that we can't make \emph{some} progress here, but we don't expect a nice answer.

Now that we're at least probably doing this right, it's not stupid to try the giant expansion (within- and between-cohort) since we think it might have a nice answer, and therefore there might be a nice trick hiding somewhere. This will be complicated, and hopefully we are sufficiently warned now.

It's also probably worth just expanding this as a square instead of a triangle, but we don't expect any insight or even substantive difference there (as of now).

\section{Bounding the between-cohort sum of squares}
For an epidemic with $B>1$, $C(t)$ is above 1 at low $t$ and ends below 1 at high $t$. We know the total number of infections is $Z$, and the weighted mean of $C(t) =1$.

To maximize the variance, we want to have a bimodal distribution, in which a fraction $\alpha$ of infections have $C(t) = 0$ and the rest have $C(t) = \frac{1}{1-\alpha}$.
In terms of sum of squares $S$, we want to maximize
$$ S = \alpha 0^2 + (1-\alpha)*(\frac{1}{1-\alpha})^2 = \frac{1}{1-\alpha}$$.
This (non-SIR-attainable) fraction is at $\alpha = 1/2$.
Thus, we know that $S \leq 2Z$

Although the SIR model won't give us a bimodal distribution for $C(t)$, we can see that $S$ grows monotonically with $B \geq 1$:

We will rely on the time scaling we've been ignoring to show this. Let's posit $B_2>B_1$ and define the scaling factor
$k = B_2/B_1$, then
$x_{B_2}(t) = x_{B_1}(kt)$ etc.

Then we have
$$C_{B_2}(t) = B_2 \int_{\tau>t}{d\tau\, x_{B_2}(\tau) \exp(t-\tau)} = kB_1\int_{\tau>t}{d\tau\, x_{B_1}(k\tau) \exp(t-\tau)}$$


We can also introduce $u = k\tau$. Note that weighting by incidence shouldn't change things because of the same scaling described above.



for $u$ near $kt$, exp(


$$C_{B_2}(t) = B_1 \int_{u>kt}{du\,x_{B_1}(u)\exp(t-u/k)}$$
$$C_{B_1}(kt) = B_1 \int_{u>kt}{du\,x_{B_1}(u)\exp(kt-u)}$$
$\exp(t-u/k) > \exp(kt-u)$, so early in the epidemic, those in the $B_2$ epidemic take advantage of a larger fraction of the time when there are susceptibles available, i.e. early, $C_{B_2}(t)>C_{B_1}(kt)$
Later on (some time after the peak), this cuts the other way: those in $B_1$ world are sick when some susceptibles remain, but in $B_2$ world they stay sick longer relative to the epidemic's dynammics, i.e. more of their infectious time is during periods with low susceptibility.

To put it in simple terms, for both B's, the mean is 1, but with $B_2$ we have higher highs and lower lows.
This should still be clear when we weight by incidence, which follows the same scaling rules as $x$ and $y$, so $S_{B_2}>S_{B_1}$
$$S_{B_1} = \frac{1}{k}\int{du\, i_{B_1}(u/k) C^2_{B_1}(uk)}$$
$$S_{B_2}  = \int{du\, i_{B_1}(u) C^2_{B_2}(u/k)} $$

So, I think we know that $S$ grows monotonically with $B \geq 1$, and that $\lim{S} B-> \infty \leq 2Z$.

\section{within-cohort variance component decreases monotonically with B}
I think we can use a similar logic to prove that the within cohort variance (or sum of squares) decreases monotonically with B. We already know that this is bounded at B ->1 at 1 (or 2Z for SS).

\end{document}
