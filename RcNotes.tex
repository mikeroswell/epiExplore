\documentclass[12pt]{article}

\begin{document}

Start with the normalized SIR:

$$ i = Bxy; \dot x = -i; \dot y = i-y $$

If $C(t)$ is the case reproductive number for an individual infected at time $t$, then the overall mean is:

$$
	\bar C 
	= \frac{\int{dt\, i(t) C(t)}}{\int{dt\, i(t)}}
	= \frac{\int{dt\, i(t) C(t)}}{Z}, 
$$

where $Z$ is the size of the epidemic.

We can calculate $C(T)$ by integrating over time since infection $\delta$:

$$C(t) = B \int{d\delta\, x(t+\delta) \exp(-\delta)}, $$

or, in terms of time of contact $\tau=t+\delta$:

$$C(t) = B \int_{\tau>t}{d\tau\, x(\tau) \exp(t-\tau)}, $$

We can also solve the $\dot y$ equation, by looking at who was infected at time $\theta$, and how many of them survived:

$$ y(t) = \int_{\theta<t}{d\theta\, i(\theta)\exp(\theta-t)},$$

and then expand:

\begin{eqnarray}
	Z \bar C
	&=& \int{dt\, i(t) C(t)}
	\\ &=& B \int{dt\, i(t) \int_{\tau>t}{d\tau\, x(\tau) \exp(t-\tau)}}
	\\ &=& B \int{d\tau\, x(\tau) \int_{t<\tau}dt\, i(t) \exp(t-\tau)}
	\\ &=& B \int{d\tau\, x(\tau) y(\tau)}
	\\ &=& \int{d\tau\, i(\tau)}
	\\ &=& Z
\end{eqnarray}

This works, but seems suspiciously hard to extend. In particular, it probably doesn't make use of the fact that we are using an exponential generation interval, whereas the hard conjecture will need to use that.

What if we try? One trick we're using here is that to integrate $f^2(x)$ over a range, we can just take twice the integral of $f(x)f(y)$ over a triangle where we assume we know $x>y$. This allows a similar limit-switching move as above.
\begin{eqnarray}
	&& \int{dt\, i(t) C^2(t)}
	\\ &=& 2B^2 \int{dt\, i(t) \int_{\sigma>t}{d\sigma\, x(\sigma) \exp(t-\sigma)}}
		\int_{\tau>\sigma}{d\tau\, x(\tau) \exp(t-\tau)}
	\\ &=& 2B^2 \int{d\tau\, x(\tau)}
		\int_{\sigma<\tau}{d\sigma\, x(\sigma)}
		\int_{t<\sigma}{dt\, i(t) \exp(t-\sigma) \exp(t-\tau)}
\end{eqnarray}

This one looks like a dead end again.

\end{document}
