\documentclass[12pt]{article}
\usepackage{amssymb}

\begin{document}

Start with the normalized SIR:

$$ i = Bxy; \dot x = -i; \dot y = i-y $$

If $C(t)$ is the expected case reproductive number for the cohort infected at time $t$, then the overall mean is:

$$
	\bar C
	= \frac{\int{dt\, i(t) C(t)}}{\int{dt\, i(t)}}
	= \frac{\int{dt\, i(t) C(t)}}{Z},
$$

where $Z$ is the size of the epidemic.

Individuals in a cohort are effectively assumed to differ only in terms of how long they are infectious. We calculate $C(t)$ as a weighted average over this duration. The probability of having duration $D$ is just $\exp(-\delta)$ in this normalized model, and the reproductive number for this group is given by:
$$C(t, D) = B \int^D{d\delta\, x(t+\delta)}, $$

Since the probability of being infectious for exactly $D$ is $\exp(-D)$:
\begin{eqnarray}
	C(t)
	&=& B \int{dD\, \exp(-D) \int^D{d\delta\, x(t+\delta)}}
	\\ &=& B \int{dD\, \exp(-D) \int_{\delta<D}{d\delta\, x(t+\delta)}}
	\\ &=& B \int{d\delta\, x(t+\delta)\int_{D>\delta}{dD\, \exp(-D) }}
	\\ &=& B \int{d\delta\, x(t+\delta) \exp(-\delta)}
\end{eqnarray}

or, in terms of time of contact $\tau=t+\delta$:
$$C(t) = B \int_{\tau>t}{d\tau\, x(\tau) \exp(t-\tau)}, $$

We can also solve the $\dot y$ equation, by looking at who was infected at time $\theta$, and how many of them survived:

$$ y(t) = \int_{\theta<t}{d\theta\, i(\theta)\exp(\theta-t)},$$

and then expand:

\begin{eqnarray}
	Z \bar C
	&=& \int{dt\, i(t) C(t)}
	\\ &=& B \int{dt\, i(t) \int_{\tau>t}{d\tau\, x(\tau) \exp(t-\tau)}}
	\\ &=& B \int{d\tau\, x(\tau) \int_{t<\tau}dt\, i(t) \exp(t-\tau)}
	\\ &=& B \int{d\tau\, x(\tau) y(\tau)}
	\\ &=& \int{d\tau\, i(\tau)}
	\\ &=& Z
\end{eqnarray}

Thus, $\bar C = 1$, as expected. [Note: we could presumably fiddle with having a finite $y_0$ by playing with integral limits in the array above to get something like $(Z-x_0)/Z$.]

Now try to expand the sum of squares. One trick we're using here is that to compute the square of the integral of $f(x)$ over a range, we can just take twice the integral of $f(x)f(y)$ over a triangle where we assume we know $x>y$.

First, we look only at between-cohort variance, so our first step is to expand the sum of squares there.

\begin{eqnarray}
\int{dt\, i(t) C^2(t)}
	\\ = \int{dt\, i(t) \left(
		B \int_{\tau>t}{d\tau\, x(\tau) \exp(t-\tau)}
	\right)^2}
	\\ = B^2 \int{dt\, i(t) \left(
		\int_{\tau>t}{d\tau\, x(\tau) \exp(t-\tau)}
	\right)^2}
	\\ =  2B^2 \int{dt\, i(t) 
		\int_{\tau>t}{d\tau\,  x(\tau) \exp(t-\tau)
			\int_{\sigma>\tau}{d\sigma\, x(\sigma)\exp(t-\sigma)}
		}
	}
	\\ =  2B^2 \int{
		d\sigma\, x(\sigma) \int_{\tau<\sigma}{
			d\tau\,  x(\tau) \int_{t<\tau}{
				dt\, i(t) \exp(t-\tau) \exp(t-\sigma) 
			}
		}
	}
\end{eqnarray}

It is not obvious how to do the inner integral, and we expect it to be complicated. It's also not obvious that we can't make \emph{some} progress here, but we don't expect a nice answer.

Now that we're at least probably doing this right, it's not stupid to try the giant expansion (within- and between-cohort) since we think it might have a nice answer, and therefore there might be a nice trick hiding somewhere. This will be complicated, and hopefully we are sufficiently warned now.

It's also probably worth just expanding this as a square instead of a triangle, but we don't expect any insight or even substantive difference there (as of now).

\end{document}
