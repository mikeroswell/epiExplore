\documentclass[12pt]{article}

\begin{document}

Start with the normalized SIR:

$$ i = Bxy; \dot x = -i; \dot y = i-y $$

If $C(t)$ is the case reproductive number for an individual infected at time $t$, then the overall mean is:

$$
	\bar C
	= \frac{\int{dt\, i(t) C(t)}}{\int{dt\, i(t)}}
	= \frac{\int{dt\, i(t) C(t)}}{Z},
$$

where $Z$ is the size of the epidemic.

We can calculate $C(T)$ by integrating over time since infection $\delta$:

$$C(t) = B \int{d\delta\, x(t+\delta) \exp(-\delta)}, $$

or, in terms of time of contact $\tau=t+\delta$:

$$C(t) = B \int_{\tau>t}{d\tau\, x(\tau) \exp(t-\tau)}, $$

It may be helpful to calculate $J(t)$, or the expected number of susceptible contacts for an individual infected at time $t$
$$J(t) = \frac{C(t)}{B} = \int_{\tau>t}{d\tau\, x(\tau) \exp(t-\tau)}, $$

We can also solve the $\dot y$ equation, by looking at who was infected at time $\theta$, and how many of them survived:

$$ y(t) = \int_{\theta<t}{d\theta\, i(\theta)\exp(\theta-t)},$$

and then expand:

\begin{eqnarray}
	Z \bar C
	&=& \int{dt\, i(t) C(t)}
	\\ &=& B \int{dt\, i(t) \int_{\tau>t}{d\tau\, x(\tau) \exp(t-\tau)}}
	\\ &=& B \int{d\tau\, x(\tau) \int_{t<\tau}dt\, i(t) \exp(t-\tau)}
	\\ &=& B \int{d\tau\, x(\tau) y(\tau)}
	\\ &=& \int{d\tau\, i(\tau)}
	\\ &=& Z
\end{eqnarray}

Now try to expand the sum of squares. One trick we're using here is that to compute the square of the integral of $f(x)$ over a range, we can just take twice the integral of $f(x)f(y)$ over a triangle where we assume we know $x>y$.

\begin{eqnarray}
\int{dt\, i(t) C^2(t)}
	\\ = \int{dt\, i(t) (B \int_{\tau>t}{d\tau\, x(\tau) \exp(t-\tau)})^2}
	\\ = B^2 \int{dt\, i(t) (\int_{\tau>t}{d\tau\, x(\tau) \exp(t-\tau)})^2}
	\\ =  2B^2 \int{dt\, i(t) \int_{\tau>t}{d\tau\, \int_{\sigma>\tau}{d\sigma\, x(\tau) x(\sigma) \exp(t-\tau)\exp(t-\sigma)}}}
\end{eqnarray}

Now, use the limit-switching move as above, then substitute $J(\tau)$
\begin{eqnarray}
2B^2 \int{dt\, i(t) \int_{\tau>t}{d\tau\, \int_{\sigma>\tau}{d\sigma\, x(\tau) x(\sigma) \exp(t-\tau)\exp(t-\sigma)}}}
  \\ = 2B^2 \int{d\tau\, y(\tau)x(\tau) \int_{\sigma > \tau}{d\sigma\, x(\sigma)\exp(\tau-\sigma)}}
  \\ = 2B^2 \int{d\tau\, y(\tau)x(\tau)J(\tau)}
\end{eqnarray}

Maybe there's something good with the expression above; following the same logic as expr 1-6, this simplifies to $2Z$ as well.

	% \\ &=& 2B \int{dt\, i(t) \int_{\tau>t}{d\tau\, x(\tau) \exp(t-\tau) C(\tau)}}
	% \\ &=& 2B \int{d\tau\, x(\tau) C(\tau) \int_{t<\tau}{dt\, i(t) \exp(t-\tau)}}
	% \\ &=& 2B \int{d\tau\, x(\tau) C(\tau) y(\tau)}
	% \\ &=& 2 \int{d\tau\, C(\tau) i(\tau)}


% Wait! Does this prove the Roswell Conjecture‽ It \emph{seems} too simple (we haven't seen how far the “proof” would go with a different infectiousness kernel), but maybe we accidentally leveraged the distribution-matching property of the exponential (infectious period has same distribution as generation interval) – that actually seems plausible.
%
% PS: This does balance correctly, fwiw: it yields a squared-sum of $2Z/Z = 2$ to go with the mean of 1, so a variance of 1.

\end{document}
