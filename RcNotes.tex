\documentclass[12pt]{article}

\begin{document}

Start with the normalized SIR:

$$ i = Bxy; \dot x = -i; \dot y = i-y $$

If $C(t)$ is the case reproductive number for an individual infected at time $t$, then the overall mean is:

$$
	\bar C 
	= \frac{\int{dt\, i(t) C(t)}}{\int{dt\, i(t)}}
	= \frac{\int{dt\, i(t) C(t)}}{Z}, 
$$

where $Z$ is the size of the epidemic.

We can integrate over time since infection $\delta$, and write:

$$C(t) = B \int{d\delta\, x(t+\delta) \exp(-\delta)}$$

If we also solve the $\dot y$ equation, by looking at who has come in $\tau$ time ago and how many of them survived, and write:

$$ y(t) = \int{d\tau\, i(t-\tau)\exp(-\tau)}$$

\emph{and} had any patience with integrals, we could presumably expand and substitute and get something like: 

$$
	Z \bar C  = B \int{dt\, x(t) y(t)} = \int{dt\, i(t)}.
$$

This seems to be what I wanted, but also seems to be a dead end. It's hard to imagine any tricks to try if we squared the whole integral for $C(t)$ and plugged \emph{that} in.

\end{document}
